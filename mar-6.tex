\documentclass[12pt]{article}

\usepackage[a4paper,margin=2cm]{geometry}
\usepackage{amsmath}
\usepackage{amssymb}
\usepackage{amsthm}
\usepackage{amsfonts}

\usepackage{algpseudocode}
\usepackage[plain]{algorithm}

\usepackage{caption}
\usepackage{graphicx}
\usepackage{float}
\usepackage{listings}
\usepackage[table]{xcolor}
\usepackage{booktabs}
\usepackage[colorlinks,allcolors=newblue]{hyperref}
\usepackage{microtype}

\usepackage{tikz}
\usepackage[framemethod=TikZ]{mdframed}
\definecolor{newblue}{rgb}{0.2,0.2,0.6}

\usepackage{listings}
\lstset{
    basicstyle=\scriptsize\ttfamily,
    commentstyle=\ttfamily\color{gray},
    numbers=left,
    numberstyle=\ttfamily\color{gray}\footnotesize,
    stepnumber=1,
    numbersep=5pt,
    backgroundcolor=\color{white},
    showspaces=false,
    showstringspaces=false,
    showtabs=false,
    frame=single,
    tabsize=2,
    captionpos=b,
    breaklines=true,
    breakatwhitespace=false,
    title=\ttfamily\lstname,
    escapeinside={},
    keywordstyle={},
    morekeywords={}
}

\newcommand{\handout}[5]{
   \renewcommand{\thepage}{#1-\arabic{page}}
   \noindent
   \begin{center}
   \framebox{
      \vbox{
    \hbox to 5.78in { {\bf ELE 585: Parallel Computing} \hfill #2 }
       \vspace{4mm}
       \hbox to 5.78in { {\Large \hfill #5  \hfill} }
       \vspace{2mm}
       \hbox to 5.78in { {\it #3 \hfill #4} }
      }
   }
   \end{center}
   \vspace*{4mm}
}

\renewcommand{\paragraph}[1]{\medskip \noindent {\bf #1.}}

\newcommand{\lecture}[3]{\handout{#1}{#2}{Professor David Wentzlaff}{By #3}{#1}}

%%%%%% FOUND AT
% https://texblog.org/2015/09/30/fancy-boxes-for-theorem-lemma-and-proof-with-mdframed/

%%%%%%%%%%%%%%%%%%%%%%%%%%%%%%
%Theorem
\newcounter{theo}[section]\setcounter{theo}{0}
\renewcommand{\thetheo}{\arabic{section}.\arabic{theo}}
\newenvironment{theo}[2][]{%
\refstepcounter{theo}%
\ifstrempty{#1}%
{\mdfsetup{%
frametitle={%
\tikz[baseline=(current bounding box.east),outer sep=0pt]
\node[anchor=east,rectangle,fill=blue!20]
{\strut Theorem~\thetheo};}}
}%
{\mdfsetup{%
frametitle={%
\tikz[baseline=(current bounding box.east),outer sep=0pt]
\node[anchor=east,rectangle,fill=blue!20]
{\strut Theorem~\thetheo:~#1};}}%
}%
\mdfsetup{innertopmargin=5pt,innerbottommargin=10pt,linecolor=blue!20,%
linewidth=2pt,topline=true,%
frametitleaboveskip=\dimexpr-\ht\strutbox\relax
}
\begin{mdframed}[]\relax%
\label{#2}}{\end{mdframed}}

%%%%%%%%%%%%%%%%%%%%%%%%%%%%%%
%Proposition
\newcounter{prp}[section]\setcounter{prp}{0}
\renewcommand{\theprp}{\arabic{section}.\arabic{prp}}
\newenvironment{prp}[2][]{%
\refstepcounter{prp}%
\ifstrempty{#1}%
{\mdfsetup{%
frametitle={%
\tikz[baseline=(current bounding box.east),outer sep=0pt]
\node[anchor=east,rectangle,fill=magenta!20]
{\strut Proposition~\theprp};}}
}%
{\mdfsetup{%
frametitle={%
\tikz[baseline=(current bounding box.east),outer sep=0pt]
\node[anchor=east,rectangle,fill=magenta!20]
{\strut Proposition~\theprp:~#1};}}%
}%
\mdfsetup{innertopmargin=5pt,innerbottommargin=10pt,linecolor=magenta!20,%
linewidth=2pt,topline=true,%
frametitleaboveskip=\dimexpr-\ht\strutbox\relax
}
\begin{mdframed}[]\relax%
\label{#2}}{\end{mdframed}}

%%%%%%%%%%%%%%%%%%%%%%%%%%%%%%
%Lemma
\newcounter{lem}[section]\setcounter{lem}{0}
\renewcommand{\thelem}{\arabic{section}.\arabic{lem}}
\newenvironment{lem}[2][]{%
\refstepcounter{lem}%
\ifstrempty{#1}%
{\mdfsetup{%
frametitle={%
\tikz[baseline=(current bounding box.east),outer sep=0pt]
\node[anchor=east,rectangle,fill=green!20]
{\strut Lemma~\thelem};}}
}%
{\mdfsetup{%
frametitle={%
\tikz[baseline=(current bounding box.east),outer sep=0pt]
\node[anchor=east,rectangle,fill=green!20]
{\strut Lemma~\thetheo:~#1};}}%
}%
\mdfsetup{innertopmargin=5pt,innerbottommargin=10pt,linecolor=green!20,%
linewidth=2pt,topline=true,%
frametitleaboveskip=\dimexpr-\ht\strutbox\relax
}
\begin{mdframed}[]\relax%
\label{#2}}{\end{mdframed}}

%%%%%%%%%%%%%%%%%%%%%%%%%%%%%%
%Proof

\newcounter{prf}[section]\setcounter{prf}{0}
\renewcommand{\theprf}{\arabic{section}.\arabic{prf}}
\newenvironment{prf}[2][]{%
\refstepcounter{prf}%
\ifstrempty{#1}%
{\mdfsetup{%
frametitle={%
\tikz[baseline=(current bounding box.east),outer sep=0pt]
\node[anchor=east,rectangle,fill=red!20]
{\strut Proof~\theprf};}}
}%
{\mdfsetup{%
frametitle={%
\tikz[baseline=(current bounding box.east),outer sep=0pt]
\node[anchor=east,rectangle,fill=red!20]
{\strut Proof~\theprf:~#1};}}%
}%
\mdfsetup{innertopmargin=5pt,innerbottommargin=10pt,linecolor=red!20,%
linewidth=2pt,topline=true,%
frametitleaboveskip=\dimexpr-\ht\strutbox\relax
}
\begin{mdframed}[]\relax%
\label{#2}}{\end{mdframed}}

%%%%%%%%%%%%%%%%%%%%%%%%%%%%%%
%Remark
\newcounter{rmk}[section]\setcounter{rmk}{0}
\renewcommand{\thermk}{\arabic{section}.\arabic{rmk}}
\newenvironment{rmk}[2][]{%
\refstepcounter{rmk}%
\ifstrempty{#1}%
{\mdfsetup{%
frametitle={%
\tikz[baseline=(current bounding box.east),outer sep=0pt]
\node[anchor=east,rectangle,fill=yellow!50]
{\strut Remark~\thermk};}}
}%
{\mdfsetup{%
frametitle={%
\tikz[baseline=(current bounding box.east),outer sep=0pt]
\node[anchor=east,rectangle,fill=yellow!50]
{\strut Remark~\thermk:~#1};}}%
}%
\mdfsetup{innertopmargin=5pt,innerbottommargin=10pt,linecolor=yellow!50,%
linewidth=2pt,topline=true,%
frametitleaboveskip=\dimexpr-\ht\strutbox\relax
}
\begin{mdframed}[]\relax%
\label{#2}}{\end{mdframed}}

%%%%%%%%%%%%%%%%%%%%%%%%%%%%%%
%Definiton
\newcounter{dfn}[section]\setcounter{dfn}{0}
\renewcommand{\thedfn}{\arabic{section}.\arabic{dfn}}
\newenvironment{dfn}[2][]{%
\refstepcounter{dfn}%
\ifstrempty{#1}%
{\mdfsetup{%
frametitle={%
\tikz[baseline=(current bounding box.east),outer sep=0pt]
\node[anchor=east,rectangle,fill=cyan!20]
{\strut Definition~\thedfn};}}
}%
{\mdfsetup{%
frametitle={%
\tikz[baseline=(current bounding box.east),outer sep=0pt]
\node[anchor=east,rectangle,fill=cyan!20]
{\strut Definition~\thedfn:~#1};}}%
}%
\mdfsetup{innertopmargin=5pt,innerbottommargin=10pt,linecolor=cyan!20,%
linewidth=2pt,topline=true,%
frametitleaboveskip=\dimexpr-\ht\strutbox\relax
}
\begin{mdframed}[]\relax%
\label{#2}}{\end{mdframed}}

%%%%%%%%%%%%%%%%%%%%%%%%%%%%%%
%Information
\newcounter{info}[section]\setcounter{info}{0}
\renewcommand{\theinfo}{\arabic{section}.\arabic{info}}
\newenvironment{info}[2][]{%
  \refstepcounter{info}%
  \ifstrempty{#1}%
  {\mdfsetup{%
  frametitle={%
  \tikz[baseline=(current bounding box.east),outer sep=0pt]
  \node[anchor=east,rectangle,fill=gray!20]
  {\strut Information~\theinfo};}}
  }%
  {\mdfsetup{%
  frametitle={%
  \tikz[baseline=(current bounding box.east),outer sep=0pt]
  \node[anchor=east,rectangle,fill=gray!20]
  {\strut Information~\theinfo:~#1};}}%
  }%
  \mdfsetup{innertopmargin=5pt,innerbottommargin=10pt,linecolor=gray!20,%
  linewidth=2pt,topline=true,%
  frametitleaboveskip=\dimexpr-\ht\strutbox\relax
  }
  \begin{mdframed}[]\relax%
  \label{#2}
}
{
  \end{mdframed}
}

%%%%%%%%%%%%%%%%%%%%%%%%%%%%%%
%Example
\newcounter{eg}[section]\setcounter{eg}{0}
\renewcommand{\theeg}{\arabic{section}.\arabic{eg}}
\newenvironment{eg}[2][]{%
  \refstepcounter{eg}%
  \ifstrempty{#1}%
  {\mdfsetup{%
  frametitle={%
  \tikz[baseline=(current bounding box.east),outer sep=0pt]
  \node[anchor=east,rectangle,fill=black!20]
  {\strut Example~\theeg};}}
  }%
  {\mdfsetup{%
  frametitle={%
  \tikz[baseline=(current bounding box.east),outer sep=0pt]
  \node[anchor=east,rectangle,fill=black!20]
  {\strut Example~\theeg:~#1};}}%
  }%
  \mdfsetup{innertopmargin=5pt,innerbottommargin=10pt,linecolor=black!20,%
  linewidth=2pt,topline=true,%
  frametitleaboveskip=\dimexpr-\ht\strutbox\relax
  }
  \begin{mdframed}[]\relax%
  \label{#2}
}
{
  \end{mdframed}
}


\definecolor{dfn-color}{rgb}{0,0.48,0.65}

% https://tex.stackexchange.com/questions/227913/meet-and-join-symbols-for-mathematical-lattice-utf-∨-and-∧
\DeclareFontFamily{U}{matha}{\hyphenchar\font45}
\DeclareFontShape{U}{matha}{m}{n}{
      <5> <6> <7> <8> <9> <10> gen * matha
      <10.95> matha10 <12> <14.4> <17.28> <20.74> <24.88> matha12
      }{}
\DeclareSymbolFont{matha}{U}{matha}{m}{n}
\DeclareMathSymbol{\meet}{2}{matha}{"5E}
\DeclareMathSymbol{\join}{2}{matha}{"5F}

\newcommand*{\QED}{\hfill\ensuremath{\square}}
\DeclareMathOperator{\lcm}{lcm}


\begin{document}
\lecture{Class Discussion Notes}{March 6, 2019}{Dev Dabke}

\section{Snoopy Protocols}

DW: We think of a snoopy protocol as one that listens to all cache operations in the system.

DW: suppose we don't have caches. What happens when you want to access data? You go to main memory. Is there a coherence issue?
SH: no coherence issue.
DW: what is the problem with this?
High latency and bandwidth of the bus to access memory is limited, so there is a bottleneck.

Write-through vs. write-back

write-back: eventually write to main memory, usually when you eject from cache
write-through: when you write your local cache, you write through to main memory
In some consistency models, it's okay that processors don't see an updated value.
Whether or not a coherence model is correct, depends on the consistency guarantee provided in hardware.
However, you can have more relaxed consistency models, and perhaps punt to software.

DW: Why are relaxed consistency models not popular in 2019?
??: it's hard to program
DW: there's no truth that certain consistency models are better; they're just different and can be useful/preferred in different contexts. It can be an aesthetic question, just based on preferences.
But, most people would say that preserving some (temporal) order is probably good.
George: this is not necessarily preferable, but it's not necessarily good that preserving (temporal) order at the hardware.
DW: there are compilers, etc., that can insert fences that can guarantee some consistency
George: I would assume that the lazy (or no so strict models) of consistency would be easier to design in terms of hardware.
DW: there is a widespread disagreement in the comp arch world about this topic. I think it's easier to do weaker consistency, but there are faculty at MIT that claimed it was easier.
The reason this became important is because there's debate on what consistency model RISC 5 should adopt.
The memory working group (led by Dan Lustig, who took this class the first time it was offered at Princeton) had a big fight about whether or not RISC 5 should allow weaker consistency models.
The outcome was a dual path that officially they use weak consistency models, BUT you can elect to implement a total store ordering (TSO), etc., and to make matters worse, the major linux vendors said that they won't ship anything unless they it's TSO.
A lot of the vendors were okay with weak consistency, especially embedded people because they don't want to deal with this hardware complexity.
You can also have different instructions (like in SPARC) that have different consistency guarantees.
By using fences, etc., you can make weaker consistency models (that are correct) by making them stronger, but there are some architectures that hilariously don't have correct implementations of weak consistency.
DW: personally, I don't think these things matter because there aren't planes falling out the sky; you can just put in fences, etc., and just make it work.
DW: in some sense, without even explicit consistency model, typically a system has some consistency model, even if it doesn't have name or is particularly strong.
DW: can we get stale data even with write-through?
GC: yes, e.g. CPU 1 loads address 5, then CPU writes to address 5, and then CPU 1 does another load, which would still get the old value because it would be in the cache.

---------------------------------------------

What is snooping?
Suppose that CPU 1 wants to write to memory; it puts out a snoopy message on the bus to let everyone know that it's going to write to memory.
It's going to broadcast the address it wants to use and the state it wants to get it into it and everyone has to listen to it.
Because of this, you have to dual-port or time-multiplex the tags on this cache.
----
SH: not sure why we need to dual-port?
DW: look at tag in the cache and we have status bits in the cache.
Possibly, two bits or something like that.
If you have the CPU as accessing the cache at the same time that another CPU is broadcasting on the bus, then you need a way to handle both at the same time.
Therefore, you need to time-multiplex or dual-port the status bits, i.e.\ the tags.
With time-muxing you could starve out your own access as the system scales with traffic.
Therefore, you need two full ports.
----
So what does this bus look like?
What do these wires look like?
Typically you have these wires with read, or intent to read, etc.\ with an address.
Usually, the address is of a cache block, not a real address.
Typically, with ``intent to write'' you wait for other people to invalidate, e.g.\ you guarantee that everyone self-updates within n cycles and if nothing happens in n cycles, you are free to proceed.
The arbitration guarantee of the bus allows for a response time.
If someone has dirty data, they are the only one who would have it, so you wait for just the one person to respond.

----
Invalidation vs. update protocols is an orthogonal concern from write-through vs. write-back. While some are more common, you can still have weird things like write-through invalidation protocol.
=========
Looking at Per's paper.
DW: Could you create a cache coherence protocol where you turn a load/store into something else?
You could have a compiler that transforms loads/stores into more complex code that implements that cache coherence protocol, i.e.\ you put cache coherence in software.
SH: hardware seems more popular?
DW: but people do this in software, like Sarita (UIUC). But hardware acceleration helps a lot.
Her stuff these days also uses hardware acceleration, but the hardware has a weak model, whereas the software model builds a stronger model.
You could use software to transform loads and stores into messages across some distributed system.
We see this in Internet-scale applications or data-center-scale applications, e.g.\ memcached where you build a distributed key-value store.
You could have loads/stores that aren't cacheable, or you just write-through on everything.
---
Another way to do this is with page coherence, which is keeping information consistent at the system level, where you could build a coherence protocol with two different connected computers.
VirtualIron (owned by VMWare now) does this.
Why doesn't this work? Granularity issue
Dfn: \textit{false sharing} this idea that if you go to access a piece of data in a cacheline or in a block and it invalidates somebody else and if you have a smaller block size and it would not have invalidated the data, then this is false sharing.


=============

Let's look at the midwest papers.

One in UIUC and the other Wisconsin. These people clearly knew about each other.
Typically Goodman and his student gets credit for snoopy cache coherence.
Janak Patel is professor at UIUC.

Jim Goodman paper
Where does snoopy cache coherence show up in this paper?
Doesn't show up in the first few pages.
What did Goodman think the main contribution?
Exploiting temporal locality for reducing bus traffic.
But Goodman's section 3 is a diamond in the rough; sometimes authors don't even realize the most important thing in their papers.
PD: Compromise between write-through and write-back as ``write-once'' is main contribution?
DW: main contribution: introduces the concept of states with the tags

\end{document}
