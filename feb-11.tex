\documentclass[12pt]{article}

\usepackage[a4paper,margin=2cm]{geometry}
\usepackage{amsmath}
\usepackage{amssymb}
\usepackage{amsthm}
\usepackage{amsfonts}

\usepackage{algpseudocode}
\usepackage[plain]{algorithm}

\usepackage{caption}
\usepackage{graphicx}
\usepackage{float}
\usepackage{listings}
\usepackage[table]{xcolor}
\usepackage{booktabs}
\usepackage[colorlinks,allcolors=newblue]{hyperref}
\usepackage{microtype}

\usepackage{tikz}
\usepackage[framemethod=TikZ]{mdframed}
\definecolor{newblue}{rgb}{0.2,0.2,0.6}

\usepackage{listings}
\lstset{
    basicstyle=\scriptsize\ttfamily,
    commentstyle=\ttfamily\color{gray},
    numbers=left,
    numberstyle=\ttfamily\color{gray}\footnotesize,
    stepnumber=1,
    numbersep=5pt,
    backgroundcolor=\color{white},
    showspaces=false,
    showstringspaces=false,
    showtabs=false,
    frame=single,
    tabsize=2,
    captionpos=b,
    breaklines=true,
    breakatwhitespace=false,
    title=\ttfamily\lstname,
    escapeinside={},
    keywordstyle={},
    morekeywords={}
}

\newcommand{\handout}[5]{
   \renewcommand{\thepage}{#1-\arabic{page}}
   \noindent
   \begin{center}
   \framebox{
      \vbox{
    \hbox to 5.78in { {\bf ELE 585: Parallel Computing} \hfill #2 }
       \vspace{4mm}
       \hbox to 5.78in { {\Large \hfill #5  \hfill} }
       \vspace{2mm}
       \hbox to 5.78in { {\it #3 \hfill #4} }
      }
   }
   \end{center}
   \vspace*{4mm}
}

\renewcommand{\paragraph}[1]{\medskip \noindent {\bf #1.}}

\newcommand{\lecture}[3]{\handout{#1}{#2}{Professor David Wentzlaff}{By #3}{#1}}

%%%%%% FOUND AT
% https://texblog.org/2015/09/30/fancy-boxes-for-theorem-lemma-and-proof-with-mdframed/

%%%%%%%%%%%%%%%%%%%%%%%%%%%%%%
%Theorem
\newcounter{theo}[section]\setcounter{theo}{0}
\renewcommand{\thetheo}{\arabic{section}.\arabic{theo}}
\newenvironment{theo}[2][]{%
\refstepcounter{theo}%
\ifstrempty{#1}%
{\mdfsetup{%
frametitle={%
\tikz[baseline=(current bounding box.east),outer sep=0pt]
\node[anchor=east,rectangle,fill=blue!20]
{\strut Theorem~\thetheo};}}
}%
{\mdfsetup{%
frametitle={%
\tikz[baseline=(current bounding box.east),outer sep=0pt]
\node[anchor=east,rectangle,fill=blue!20]
{\strut Theorem~\thetheo:~#1};}}%
}%
\mdfsetup{innertopmargin=5pt,innerbottommargin=10pt,linecolor=blue!20,%
linewidth=2pt,topline=true,%
frametitleaboveskip=\dimexpr-\ht\strutbox\relax
}
\begin{mdframed}[]\relax%
\label{#2}}{\end{mdframed}}

%%%%%%%%%%%%%%%%%%%%%%%%%%%%%%
%Proposition
\newcounter{prp}[section]\setcounter{prp}{0}
\renewcommand{\theprp}{\arabic{section}.\arabic{prp}}
\newenvironment{prp}[2][]{%
\refstepcounter{prp}%
\ifstrempty{#1}%
{\mdfsetup{%
frametitle={%
\tikz[baseline=(current bounding box.east),outer sep=0pt]
\node[anchor=east,rectangle,fill=magenta!20]
{\strut Proposition~\theprp};}}
}%
{\mdfsetup{%
frametitle={%
\tikz[baseline=(current bounding box.east),outer sep=0pt]
\node[anchor=east,rectangle,fill=magenta!20]
{\strut Proposition~\theprp:~#1};}}%
}%
\mdfsetup{innertopmargin=5pt,innerbottommargin=10pt,linecolor=magenta!20,%
linewidth=2pt,topline=true,%
frametitleaboveskip=\dimexpr-\ht\strutbox\relax
}
\begin{mdframed}[]\relax%
\label{#2}}{\end{mdframed}}

%%%%%%%%%%%%%%%%%%%%%%%%%%%%%%
%Lemma
\newcounter{lem}[section]\setcounter{lem}{0}
\renewcommand{\thelem}{\arabic{section}.\arabic{lem}}
\newenvironment{lem}[2][]{%
\refstepcounter{lem}%
\ifstrempty{#1}%
{\mdfsetup{%
frametitle={%
\tikz[baseline=(current bounding box.east),outer sep=0pt]
\node[anchor=east,rectangle,fill=green!20]
{\strut Lemma~\thelem};}}
}%
{\mdfsetup{%
frametitle={%
\tikz[baseline=(current bounding box.east),outer sep=0pt]
\node[anchor=east,rectangle,fill=green!20]
{\strut Lemma~\thetheo:~#1};}}%
}%
\mdfsetup{innertopmargin=5pt,innerbottommargin=10pt,linecolor=green!20,%
linewidth=2pt,topline=true,%
frametitleaboveskip=\dimexpr-\ht\strutbox\relax
}
\begin{mdframed}[]\relax%
\label{#2}}{\end{mdframed}}

%%%%%%%%%%%%%%%%%%%%%%%%%%%%%%
%Proof

\newcounter{prf}[section]\setcounter{prf}{0}
\renewcommand{\theprf}{\arabic{section}.\arabic{prf}}
\newenvironment{prf}[2][]{%
\refstepcounter{prf}%
\ifstrempty{#1}%
{\mdfsetup{%
frametitle={%
\tikz[baseline=(current bounding box.east),outer sep=0pt]
\node[anchor=east,rectangle,fill=red!20]
{\strut Proof~\theprf};}}
}%
{\mdfsetup{%
frametitle={%
\tikz[baseline=(current bounding box.east),outer sep=0pt]
\node[anchor=east,rectangle,fill=red!20]
{\strut Proof~\theprf:~#1};}}%
}%
\mdfsetup{innertopmargin=5pt,innerbottommargin=10pt,linecolor=red!20,%
linewidth=2pt,topline=true,%
frametitleaboveskip=\dimexpr-\ht\strutbox\relax
}
\begin{mdframed}[]\relax%
\label{#2}}{\end{mdframed}}

%%%%%%%%%%%%%%%%%%%%%%%%%%%%%%
%Remark
\newcounter{rmk}[section]\setcounter{rmk}{0}
\renewcommand{\thermk}{\arabic{section}.\arabic{rmk}}
\newenvironment{rmk}[2][]{%
\refstepcounter{rmk}%
\ifstrempty{#1}%
{\mdfsetup{%
frametitle={%
\tikz[baseline=(current bounding box.east),outer sep=0pt]
\node[anchor=east,rectangle,fill=yellow!50]
{\strut Remark~\thermk};}}
}%
{\mdfsetup{%
frametitle={%
\tikz[baseline=(current bounding box.east),outer sep=0pt]
\node[anchor=east,rectangle,fill=yellow!50]
{\strut Remark~\thermk:~#1};}}%
}%
\mdfsetup{innertopmargin=5pt,innerbottommargin=10pt,linecolor=yellow!50,%
linewidth=2pt,topline=true,%
frametitleaboveskip=\dimexpr-\ht\strutbox\relax
}
\begin{mdframed}[]\relax%
\label{#2}}{\end{mdframed}}

%%%%%%%%%%%%%%%%%%%%%%%%%%%%%%
%Definiton
\newcounter{dfn}[section]\setcounter{dfn}{0}
\renewcommand{\thedfn}{\arabic{section}.\arabic{dfn}}
\newenvironment{dfn}[2][]{%
\refstepcounter{dfn}%
\ifstrempty{#1}%
{\mdfsetup{%
frametitle={%
\tikz[baseline=(current bounding box.east),outer sep=0pt]
\node[anchor=east,rectangle,fill=cyan!20]
{\strut Definition~\thedfn};}}
}%
{\mdfsetup{%
frametitle={%
\tikz[baseline=(current bounding box.east),outer sep=0pt]
\node[anchor=east,rectangle,fill=cyan!20]
{\strut Definition~\thedfn:~#1};}}%
}%
\mdfsetup{innertopmargin=5pt,innerbottommargin=10pt,linecolor=cyan!20,%
linewidth=2pt,topline=true,%
frametitleaboveskip=\dimexpr-\ht\strutbox\relax
}
\begin{mdframed}[]\relax%
\label{#2}}{\end{mdframed}}

%%%%%%%%%%%%%%%%%%%%%%%%%%%%%%
%Information
\newcounter{info}[section]\setcounter{info}{0}
\renewcommand{\theinfo}{\arabic{section}.\arabic{info}}
\newenvironment{info}[2][]{%
  \refstepcounter{info}%
  \ifstrempty{#1}%
  {\mdfsetup{%
  frametitle={%
  \tikz[baseline=(current bounding box.east),outer sep=0pt]
  \node[anchor=east,rectangle,fill=gray!20]
  {\strut Information~\theinfo};}}
  }%
  {\mdfsetup{%
  frametitle={%
  \tikz[baseline=(current bounding box.east),outer sep=0pt]
  \node[anchor=east,rectangle,fill=gray!20]
  {\strut Information~\theinfo:~#1};}}%
  }%
  \mdfsetup{innertopmargin=5pt,innerbottommargin=10pt,linecolor=gray!20,%
  linewidth=2pt,topline=true,%
  frametitleaboveskip=\dimexpr-\ht\strutbox\relax
  }
  \begin{mdframed}[]\relax%
  \label{#2}
}
{
  \end{mdframed}
}

%%%%%%%%%%%%%%%%%%%%%%%%%%%%%%
%Example
\newcounter{eg}[section]\setcounter{eg}{0}
\renewcommand{\theeg}{\arabic{section}.\arabic{eg}}
\newenvironment{eg}[2][]{%
  \refstepcounter{eg}%
  \ifstrempty{#1}%
  {\mdfsetup{%
  frametitle={%
  \tikz[baseline=(current bounding box.east),outer sep=0pt]
  \node[anchor=east,rectangle,fill=black!20]
  {\strut Example~\theeg};}}
  }%
  {\mdfsetup{%
  frametitle={%
  \tikz[baseline=(current bounding box.east),outer sep=0pt]
  \node[anchor=east,rectangle,fill=black!20]
  {\strut Example~\theeg:~#1};}}%
  }%
  \mdfsetup{innertopmargin=5pt,innerbottommargin=10pt,linecolor=black!20,%
  linewidth=2pt,topline=true,%
  frametitleaboveskip=\dimexpr-\ht\strutbox\relax
  }
  \begin{mdframed}[]\relax%
  \label{#2}
}
{
  \end{mdframed}
}


\definecolor{dfn-color}{rgb}{0,0.48,0.65}

% https://tex.stackexchange.com/questions/227913/meet-and-join-symbols-for-mathematical-lattice-utf-∨-and-∧
\DeclareFontFamily{U}{matha}{\hyphenchar\font45}
\DeclareFontShape{U}{matha}{m}{n}{
      <5> <6> <7> <8> <9> <10> gen * matha
      <10.95> matha10 <12> <14.4> <17.28> <20.74> <24.88> matha12
      }{}
\DeclareSymbolFont{matha}{U}{matha}{m}{n}
\DeclareMathSymbol{\meet}{2}{matha}{"5E}
\DeclareMathSymbol{\join}{2}{matha}{"5F}

\newcommand*{\QED}{\hfill\ensuremath{\square}}
\DeclareMathOperator{\lcm}{lcm}


\begin{document}
\lecture{Class Discussion Notes}{February 11, 2019}{Dev Dabke}

\section{Overview}\label{feb-11:overview}
On this day, we primarily discussed the paper \textit{A View from Berkeley}\cite{Asanovic:EECS-2006-183} (though we also read \textit{Classical Iterative Methods for Systems of Linear Equations} from \textit{Parallel and Distributed Computation: Numerical Methods}\cite{Bertsekas:1989:PDC:59912}, but did not have time to discuss it).

\section{A View from Berkeley}\label{feb-11:a-view}
\subsection{Paper Overview}\label{feb-11:a-view:overview}
According to our class, this paper is about
\begin{itemize}
    \item Everything about parallel computing
    \item The state of the art and how we should transform the field
\end{itemize}
This paper is authored by various researchers and scientists from various fields.
They are mostly faculty or senior research scientists at the UC Berkeley (at the time, Krste was moving from MIT to UCB).
\\ \\
Additionally, we determined that, stylistically, this paper:
\begin{itemize}
    \item Is a summary of these researchers' discussions on parallel computing
    \item A technical report
    \item A position paper
    \item Tried to lay down a roadmap of how they expect people should develop parallel programs, systems, and architectures
    \item Has a lot of strong positions
    \item Has positions based on reasoning, rather than empirical evidence (e.g.\ the argument that we should go from multicore to manycore)
\end{itemize}

\subsection{Primary Contribution}\label{feb-11:a-view:contribution}
Some ideas for this paper's primary contribution were:
\begin{itemize}
    \item Setting standards for evaluating parallel computing
    \item Providing guidelines for parallel computing
    \item The new conventional wisdoms
    \item The dwarfs (i.e.\ high-level abstraction over measuring performance for parallel computing)
    \item % TODO: add what [Dev] said
    \item % TODO: add what [?] said
    \item Though it has many contributions, it provides the most comprehensive synthesis and analysis of the best available information at the time
\end{itemize}

\subsection{Key Insights}\label{feb-11:a-view:insights}

We also enumerated what some of the key insights were
\begin{enumerate}
    \item Manycore will beat multicore
    \item The dwarfs (i.e.\ high-level abstraction over measuring performance for parallel computing)
    \item The new conventional wisdoms
    \item The potpourri of ancillary considerations (autotuners, architectural suggestions)
    \item That smaller/simpler is better: deep pipelines don't help parallelism (e.g.\ Intel Tejas processor, internally the Pentium 8, hit the \textit{power wall} and was scaled back because of power considerations)
\end{enumerate}

\subsection{Manycore vs.\ Multcore}\label{feb-11:a-view:many-vs-multi}
As an aside, we also discussed the difference between manycore and multicore, since these terms were used, but not explicitly defined, in the paper.

\rowcolors{2}{gray!25}{white}
\begin{center}
    \begin{tabular}{p{7cm} p{7cm}}
        \toprule
        Manycore & Multicore \\
        \midrule
        Bus & No C \\
        \( <10 \) cores & \( >10 \) cores \\
        Designed using processor for uniprocessor systems that you connect together & Designed from the ground up to handle a large number of cores; other things you add into the system that facilitate parallel computing \\
        \bottomrule
    \end{tabular}
\end{center}

\subsection{Like and Dislike}\label{feb-11:a-view:like-dislike}
We also discussed what everyone liked and disliked about this paper:

\begin{itemize}
    \item Sam: part about the conventional wisdom, some of the changes in how people viewed computer systems; decent amount of technical expertise required; a little lost (?); DW: look across the list and see people who things across the board (arch, supercomputing, autotuners, algo, etc.)
    \item Peter: like different tensions in different places, software: visibility/opacity (even before mobile computing); DW: before the iPhone; didn't like how vauge they were about some of the programming stuff, like human factors in productivity; DW: if you do human factors research, you'll get better programming languages (?)
    \item Gaurav: ?
    \item ???: they didn't say better arch is better; only if programmers can exploit it is it useful; DW: we should make our arch.es human focussed; ???: we should make it so we can exploit the parallelism; ???: dislike that they don't leave room for conversation; DW: this paper is very opinionated
    \item ?: very comprehensive, so great to see parts that I should consider when designing something; dislike that there's unstated assumption, probably like still using silicon; DW: manycore was drastically different; ?: doesn't discuss drasticaly drastically different
    \item ??: like that they provide a global view on what should be solved in parallel computing systems because they have experts in many different fields; we can understand how the different parts interact with each other; dislike how optimistic/pessimistic about certain things; no one uses 1000s of cores; DW: how many cores does your GPU have?; DW: their dwarves are not general/retail computing, not constrained to traditional desktop workloads; ??: was GPU under development?; DW: they talk about something called cg, predated cuda (general purpose GPUs); GPGPUs, GPUs in cg that Microsoft came up with (no control flow, to do vertex shading, textures, etc.); predates cuda, ML, etc.; rocket scientists / supercomputing people were using this; ??: manycore already built and commercialized; DW: GPGPUs had 10ish cores, etc.
    \item ?!: like that there's a summary of applications, difficulty of extending Moore's Law, but provide way to specialize on certain applications; you have to gain perf from somewhere, where are we really getting the perf?; DW: the paper is actively not discussing specialization; ?!: if you put our benchmarks, people will start to specialize on the benchmarks; DW: does this paper talk about HLS, etc., for applications? ?!: talks more about parallelization; DW: main goal is to talk about parallelization; dislike: not much, perhaps missing certain computing applications?
    \item DD: % TODO: fill out
        DW: they push their own research agendas; points that correspond to each faculty member
    \item Greg: like that the paper is systematic, enumerations, etc.; dwarfs are a huge approach to try to summarize computational patterns; don't agree that they push their research agendas; they have to propose something, so their ideas are good to consider; provide jumping off point
\end{itemize}

\subsection{Conventional Wisdom}\label{feb-11:a-view:conventional}
In some sense, this paper, in its view of the Golden Gate Bridge from UC Berkeley, is trying to give something to every researcher.
This paper thus introduces some ``conventional wisdom.''
We discuss how this ``wisdom'' has or has not held up, and if it was even true at the time.

\begin{enumerate}
    \item not really true: power vs.\ transistor cost; now everything is expensive and we've hit the Moore's Law Wall
    \item not really true, as good engineering fixed static energy problem
    \item true: errors def got worst
    \item true: constraints become more important as you shrink feature size
    \item not really true: RAMP never really took off
    \item kinda true: starting to see problems with bandwidth, but still going up
    \item old CW not even true then: Memory wall is still a problem
    \item true
    \item true
    \item true
    \item true
    \item true for some applications
\end{enumerate}

\subsection{The Dwarfs}\label{feb-11:a-view:dwarfs}

This paper also introduces a notion of \textit{dwarfs}.
Although this paper is not the first to use this terminology, they co-opted and extended it for the purposes of parallel computing.
\\ \\
We postulated a few definitions for a dwarf:
\begin{itemize}
    \item better word ``kernel''
    \item a testing program
    \item a benchmark
    \item an application domain
\end{itemize}

We converged on these distinctions:
\begin{enumerate}
    \item A \textit{dwarf} is a general application domain that is not actionable, e.g.\ Spectral Methods
    \item A \textit{kernel} is a specific problem or algorithm within a dwarf, e.g.\ FFT
    \item A \textit{benchmark} is a specific implementation of a kernel, e.g.\ a specific program that runs FFT
\end{enumerate}

We also debated if dwarfs were useful.

% TODO: -- Dev comment on utility of dwarfs
Peter: if I were to implement a kernel, I would track a dwarf to give a general gist of perf for a dwarf
% TODO: -- Dev comment on dwarfs
Sam: fair to say that Dwarfs are not actionable, but kernels or benchmarks?
Sam: what's the main drawback?
DW: we can't compare
Gaurav: doesn't make sense to compare across dwarfs
DW: how do I compare my GPU against a manycore?; interesting viewpoint that they took; decent cut through different application spaces that are roughly important; talk about Intel benchmark suite and how that maps onto the dwarfs; but the argument here is that none of this is actionable or useful

DW: let's walk through a few of the dwarfs; is this all of the applications?; they added 3 more dwarfs
\begin{enumerate}
    \item Dense LA probably useful; we will talk about communication patterns; think about how much parallelism exists in the application; mapping to topologies, but you can't map all small neighborhoods next to each other
    \item Sparse LA, localization, but further communication as well
    \item Spectral, FFT, real/complex-valued, etc.; communication from everybody to everybody; not as bad as \( n \)-body, as there is some structure [Diagram 2]
    \item \( n \)-body problem; if we zoom in on this communication pattern, but we don't see that everything communicates, so we approximate \( n \)-body with neighborhoods; good approx for some things, but not for other things
    \item Structured grids, some PDE solvers
    \item Unstructured grids, graph computations
    \item Monte Carlo, clump all embarrassingly parallel; DW: wouldn't necessary glom all embarrassingly parallel models into Monte Carlo
\end{enumerate}

\bibliography{citation}{}
\bibliographystyle{alpha}

\end{document}
