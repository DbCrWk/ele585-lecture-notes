\documentclass[12pt]{article}

\usepackage[a4paper,margin=2cm]{geometry}
\usepackage{amsmath}
\usepackage{amssymb}
\usepackage{amsthm}
\usepackage{amsfonts}

\usepackage{algpseudocode}
\usepackage[plain]{algorithm}

\usepackage{caption}
\usepackage{graphicx}
\usepackage{float}
\usepackage{listings}
\usepackage[table]{xcolor}
\usepackage{booktabs}
\usepackage[colorlinks,allcolors=newblue]{hyperref}
\usepackage{microtype}

\usepackage{tikz}
\usepackage[framemethod=TikZ]{mdframed}
\definecolor{newblue}{rgb}{0.2,0.2,0.6}

\usepackage{listings}
\lstset{
    basicstyle=\scriptsize\ttfamily,
    commentstyle=\ttfamily\color{gray},
    numbers=left,
    numberstyle=\ttfamily\color{gray}\footnotesize,
    stepnumber=1,
    numbersep=5pt,
    backgroundcolor=\color{white},
    showspaces=false,
    showstringspaces=false,
    showtabs=false,
    frame=single,
    tabsize=2,
    captionpos=b,
    breaklines=true,
    breakatwhitespace=false,
    title=\ttfamily\lstname,
    escapeinside={},
    keywordstyle={},
    morekeywords={}
}

\newcommand{\handout}[5]{
   \renewcommand{\thepage}{#1-\arabic{page}}
   \noindent
   \begin{center}
   \framebox{
      \vbox{
    \hbox to 5.78in { {\bf ELE 585: Parallel Computing} \hfill #2 }
       \vspace{4mm}
       \hbox to 5.78in { {\Large \hfill #5  \hfill} }
       \vspace{2mm}
       \hbox to 5.78in { {\it #3 \hfill #4} }
      }
   }
   \end{center}
   \vspace*{4mm}
}

\renewcommand{\paragraph}[1]{\medskip \noindent {\bf #1.}}

\newcommand{\lecture}[3]{\handout{#1}{#2}{Professor David Wentzlaff}{By #3}{#1}}

\input{config/math-boxes}

\definecolor{dfn-color}{rgb}{0,0.48,0.65}

% https://tex.stackexchange.com/questions/227913/meet-and-join-symbols-for-mathematical-lattice-utf-∨-and-∧
\DeclareFontFamily{U}{matha}{\hyphenchar\font45}
\DeclareFontShape{U}{matha}{m}{n}{
      <5> <6> <7> <8> <9> <10> gen * matha
      <10.95> matha10 <12> <14.4> <17.28> <20.74> <24.88> matha12
      }{}
\DeclareSymbolFont{matha}{U}{matha}{m}{n}
\DeclareMathSymbol{\meet}{2}{matha}{"5E}
\DeclareMathSymbol{\join}{2}{matha}{"5F}

\newcommand*{\QED}{\hfill\ensuremath{\square}}
\DeclareMathOperator{\lcm}{lcm}


\begin{document}
\lecture{Lecture Notes}{February 11, 2019}{Dev Dabke}

Discuss two papers today.

\section{A View From Berkeley}
DW: What is this paper about?
Everything about paralle computing

DW: other opinions?
G: start of the art and how we should transform the field

DW: who are the authors of this paper?
scientists from various fields

DW: are these researchers / grad students?
These are all faculty or senior research scientists are UCB (Krste moving from MIT to UCB)

DW: is this a research paper?
Summary of the discussion?
Technical report
DW: tech report can be anything
DW: is this trying to just let us know what's going on
It's trying to lay a roadmap of how they expect people should develop things
DW: is this a position paper?
Greg: def not a roadmap
Sam: could be a position paper because a lot of strong opinions

Sam: some of the statements, like we should go from multicore to manycore is based on reasoning, rather than empirical data

DW: closer to position paper (trying to push a view onto us, lit. a ``view'' from UCB)

DW: primary contribution
Peter: set some standards for evaluating parallel computing
?: guideline
DW: cw or conclusion
?: divided into many parts and each part has their own contribution

% TODO
DD: -----
?: -----

Greg: many contribution, best anaylsis of avaiable informaiton at the time

DW: what are the key insights, let's enumerate them
?: what aspects should I consider
\begin{enumerate}
    \item ??: many core will beat multi core (what is multicore vs. many core; multicore is > 1 core, manycore is)
    \item DW: dwarves
    \item DW: conventional wisdom
    \item DW: potpourri of considerations (autotuners, architectural suggestions)
    \item DW: smaller/simpler is better, for parallelism it doesn't really help to have deep pipelines; Intel tejas processor (internally Pentium 8), power wall; scaled back because of power considerations
\end{enumerate}

Compare many core vs multicore

multicore
\begin{enumerate}
    \item Bus
    \item < 10
    \item DW: designed using processor for uniprocessor systems that you connect together
    \item [Diagram 1]
\end{enumerate}

manycore
\begin{enumerate}
    \item No C
    \item > 10
    \item DW: designed from the ground up to handle a large number of cores; other things you add into the system that facilitate parallel computing
\end{enumerate}

DW: what did people like and dislike?
\begin{enumerate}
    \item Sam: part about the conventional wisdom, some of the changes in how people viewed computer systems; decent amount of technical expertise required; a little lost (?); DW: look across the list and see people who things across the board (arch, supercomputing, autotuners, algo, etc.)
    \item Peter: like different tensions in different places, software: visibility/opacity (even before mobile computing); DW: before the iPhone; didn't like how vauge they were about some of the programming stuff, like human factors in productivity; DW: if you do human factors research, you'll get better programming languages (?)
    \item Gaurav: ?
    \item ???: they didn't say better arch is better; only if programmers can exploit it is it useful; DW: we should make our arch.es human focussed; ???: we should make it so we can exploit the parallelism; ???: dislike that they don't leave room for conversation; DW: this paper is very opinionated
    \item ?: very comprehensive, so great to see parts that I should consider when designing something; dislike that there's unstated assumption, probably like still using silicon; DW: manycore was drastically different; ?: doesn't discuss drasticaly drastically different
    \item ??: like that they provide a global view on what should be solved in parallel computing systems because they have experts in many different fields; we can understand how the different parts interact with each other; dislike how optimistic/pessimistic about certain things; no one uses 1000s of cores; DW: how many cores does your GPU have?; DW: their dwarves are not general/retail computing, not constrained to traditional desktop workloads; ??: was GPU under development?; DW: they talk about something called cg, predated cuda (general purpose GPUs); GPGPUs, GPUs in cg that Microsoft came up with (no control flow, to do vertex shading, textures, etc.); predates cuda, ML, etc.; rocket scientists / supercomputing people were using this; ??: manycore already built and commercialized; DW: GPGPUs had 10ish cores, etc.
    \item ?!: like that there's a summary of applications, difficulty of extending Moore's Law, but provide way to specialize on certain applications; you have to gain perf from somewhere, where are we really getting the perf?; DW: the paper is actively not discussing specialization; ?!: if you put our benchmarks, people will start to specialize on the benchmarks; DW: does this paper talk about HLS, etc., for applications? ?!: talks more about parallelization; DW: main goal is to talk about parallelization; dislike: not much, perhaps missing certain computing applications?
    \item DD: % TODO: fill out
        DW: they push their own research agendas; points that correspond to each faculty member
    \item Greg: like that the paper is systematic, enumerations, etc.; dwarfs are a huge approach to try to summarize computational patterns; don't agree that they push their research agendas; they have to propose something, so their ideas are good to consider; provide jumping off point
\end{enumerate}

Jump into the meat of the paper:
Goes through conventional wisdoms
View of GGB from UCB, trying to give something to every researcher

CW: agree/disagree
\begin{enumerate}
    \item not really true, power vs. transistor cost
    Greg: now everything is expensive
    Sam: Moore's Law Wall
    \item not really true, dynamic/static leakage, engineering fixed static energy problem
    \item true, errors: errors def got worst
    \item true, constraints as you shrink feature size become more important
    \item not really true, RAMP never really took off
    \item kinda true, starting to see problems with bandwidth, but still going up
    \item old CW not even true then; Memory wall is still a problem
    \item true
    \item true
    \item true
    \item true
    \item true for some applications, but maybe not without constraints
\end{enumerate}

DW: what is a dwarf?
Peter: better word ``kernel''

DW: what is a benchmark?
??: testing program

DW: FFT is a kernel, is that a dwarf?
Benchmark: actual program you run
Kernel: an algo/program
Dwarf: more general
% TODO: -- Dev comment on utility of dwarfs
Peter: if I were to implement a kernel, I would track a dwarf to give a general gist of perf for a dwarf
% TODO: -- Dev comment on dwarfs
Sam: fair to say that Dwarfs are not actionable, but kernels or benchmarks?
Sam: what's the main drawback?
DW: we can't compare
Gaurav: doesn't make sense to compare across dwarfs
DW: how do I compare my GPU against a manycore?; interesting viewpoint that they took; decent cut through different application spaces that are roughly important; talk about Intel benchmark suite and how that maps onto the dwarfs; but the argument here is that none of this is actionable or useful

DW: let's walk through a few of the dwarfs; is this all of the applications?; they added 3 more dwarfs
\begin{enumerate}
    \item Dense LA probably useful; we will talk about communication patterns; think about how much parallelism exists in the application; mapping to topologies, but you can't map all small neighborhoods next to each other
    \item Sparse LA, localization, but further communication as well
    \item Spectral, FFT, real/complex-valued, etc.; communication from everybody to everybody; not as bad as \( n \)-body, as there is some structure [Diagram 2]
    \item \( n \)-body problem; if we zoom in on this communication pattern, but we don't see that everything communicates, so we approximate \( n \)-body with neighborhoods; good approx for some things, but not for other things
    \item Structured grids, some PDE solvers
    \item Unstructured grids, graph computations
    \item Monte Carlo, clump all embarrassingly parallel; DW: wouldn't necessary glom all embarrassingly parallel models into Monte Carlo
\end{enumerate}

\end{document}
