\documentclass[12pt]{article}

\usepackage[a4paper,margin=2cm]{geometry}
\usepackage{amsmath}
\usepackage{amssymb}
\usepackage{amsthm}
\usepackage{amsfonts}

\usepackage{algpseudocode}
\usepackage[plain]{algorithm}

\usepackage{caption}
\usepackage{graphicx}
\usepackage{float}
\usepackage{listings}
\usepackage[table]{xcolor}
\usepackage{booktabs}
\usepackage[colorlinks,allcolors=newblue]{hyperref}
\usepackage{microtype}

\usepackage{tikz}
\usepackage[framemethod=TikZ]{mdframed}
\definecolor{newblue}{rgb}{0.2,0.2,0.6}

\usepackage{listings}
\lstset{
    basicstyle=\scriptsize\ttfamily,
    commentstyle=\ttfamily\color{gray},
    numbers=left,
    numberstyle=\ttfamily\color{gray}\footnotesize,
    stepnumber=1,
    numbersep=5pt,
    backgroundcolor=\color{white},
    showspaces=false,
    showstringspaces=false,
    showtabs=false,
    frame=single,
    tabsize=2,
    captionpos=b,
    breaklines=true,
    breakatwhitespace=false,
    title=\ttfamily\lstname,
    escapeinside={},
    keywordstyle={},
    morekeywords={}
}

\newcommand{\handout}[5]{
   \renewcommand{\thepage}{#1-\arabic{page}}
   \noindent
   \begin{center}
   \framebox{
      \vbox{
    \hbox to 5.78in { {\bf ELE 585: Parallel Computing} \hfill #2 }
       \vspace{4mm}
       \hbox to 5.78in { {\Large \hfill #5  \hfill} }
       \vspace{2mm}
       \hbox to 5.78in { {\it #3 \hfill #4} }
      }
   }
   \end{center}
   \vspace*{4mm}
}

\renewcommand{\paragraph}[1]{\medskip \noindent {\bf #1.}}

\newcommand{\lecture}[3]{\handout{#1}{#2}{Professor David Wentzlaff}{By #3}{#1}}

%%%%%% FOUND AT
% https://texblog.org/2015/09/30/fancy-boxes-for-theorem-lemma-and-proof-with-mdframed/

%%%%%%%%%%%%%%%%%%%%%%%%%%%%%%
%Theorem
\newcounter{theo}[section]\setcounter{theo}{0}
\renewcommand{\thetheo}{\arabic{section}.\arabic{theo}}
\newenvironment{theo}[2][]{%
\refstepcounter{theo}%
\ifstrempty{#1}%
{\mdfsetup{%
frametitle={%
\tikz[baseline=(current bounding box.east),outer sep=0pt]
\node[anchor=east,rectangle,fill=blue!20]
{\strut Theorem~\thetheo};}}
}%
{\mdfsetup{%
frametitle={%
\tikz[baseline=(current bounding box.east),outer sep=0pt]
\node[anchor=east,rectangle,fill=blue!20]
{\strut Theorem~\thetheo:~#1};}}%
}%
\mdfsetup{innertopmargin=5pt,innerbottommargin=10pt,linecolor=blue!20,%
linewidth=2pt,topline=true,%
frametitleaboveskip=\dimexpr-\ht\strutbox\relax
}
\begin{mdframed}[]\relax%
\label{#2}}{\end{mdframed}}

%%%%%%%%%%%%%%%%%%%%%%%%%%%%%%
%Proposition
\newcounter{prp}[section]\setcounter{prp}{0}
\renewcommand{\theprp}{\arabic{section}.\arabic{prp}}
\newenvironment{prp}[2][]{%
\refstepcounter{prp}%
\ifstrempty{#1}%
{\mdfsetup{%
frametitle={%
\tikz[baseline=(current bounding box.east),outer sep=0pt]
\node[anchor=east,rectangle,fill=magenta!20]
{\strut Proposition~\theprp};}}
}%
{\mdfsetup{%
frametitle={%
\tikz[baseline=(current bounding box.east),outer sep=0pt]
\node[anchor=east,rectangle,fill=magenta!20]
{\strut Proposition~\theprp:~#1};}}%
}%
\mdfsetup{innertopmargin=5pt,innerbottommargin=10pt,linecolor=magenta!20,%
linewidth=2pt,topline=true,%
frametitleaboveskip=\dimexpr-\ht\strutbox\relax
}
\begin{mdframed}[]\relax%
\label{#2}}{\end{mdframed}}

%%%%%%%%%%%%%%%%%%%%%%%%%%%%%%
%Lemma
\newcounter{lem}[section]\setcounter{lem}{0}
\renewcommand{\thelem}{\arabic{section}.\arabic{lem}}
\newenvironment{lem}[2][]{%
\refstepcounter{lem}%
\ifstrempty{#1}%
{\mdfsetup{%
frametitle={%
\tikz[baseline=(current bounding box.east),outer sep=0pt]
\node[anchor=east,rectangle,fill=green!20]
{\strut Lemma~\thelem};}}
}%
{\mdfsetup{%
frametitle={%
\tikz[baseline=(current bounding box.east),outer sep=0pt]
\node[anchor=east,rectangle,fill=green!20]
{\strut Lemma~\thetheo:~#1};}}%
}%
\mdfsetup{innertopmargin=5pt,innerbottommargin=10pt,linecolor=green!20,%
linewidth=2pt,topline=true,%
frametitleaboveskip=\dimexpr-\ht\strutbox\relax
}
\begin{mdframed}[]\relax%
\label{#2}}{\end{mdframed}}

%%%%%%%%%%%%%%%%%%%%%%%%%%%%%%
%Proof

\newcounter{prf}[section]\setcounter{prf}{0}
\renewcommand{\theprf}{\arabic{section}.\arabic{prf}}
\newenvironment{prf}[2][]{%
\refstepcounter{prf}%
\ifstrempty{#1}%
{\mdfsetup{%
frametitle={%
\tikz[baseline=(current bounding box.east),outer sep=0pt]
\node[anchor=east,rectangle,fill=red!20]
{\strut Proof~\theprf};}}
}%
{\mdfsetup{%
frametitle={%
\tikz[baseline=(current bounding box.east),outer sep=0pt]
\node[anchor=east,rectangle,fill=red!20]
{\strut Proof~\theprf:~#1};}}%
}%
\mdfsetup{innertopmargin=5pt,innerbottommargin=10pt,linecolor=red!20,%
linewidth=2pt,topline=true,%
frametitleaboveskip=\dimexpr-\ht\strutbox\relax
}
\begin{mdframed}[]\relax%
\label{#2}}{\end{mdframed}}

%%%%%%%%%%%%%%%%%%%%%%%%%%%%%%
%Remark
\newcounter{rmk}[section]\setcounter{rmk}{0}
\renewcommand{\thermk}{\arabic{section}.\arabic{rmk}}
\newenvironment{rmk}[2][]{%
\refstepcounter{rmk}%
\ifstrempty{#1}%
{\mdfsetup{%
frametitle={%
\tikz[baseline=(current bounding box.east),outer sep=0pt]
\node[anchor=east,rectangle,fill=yellow!50]
{\strut Remark~\thermk};}}
}%
{\mdfsetup{%
frametitle={%
\tikz[baseline=(current bounding box.east),outer sep=0pt]
\node[anchor=east,rectangle,fill=yellow!50]
{\strut Remark~\thermk:~#1};}}%
}%
\mdfsetup{innertopmargin=5pt,innerbottommargin=10pt,linecolor=yellow!50,%
linewidth=2pt,topline=true,%
frametitleaboveskip=\dimexpr-\ht\strutbox\relax
}
\begin{mdframed}[]\relax%
\label{#2}}{\end{mdframed}}

%%%%%%%%%%%%%%%%%%%%%%%%%%%%%%
%Definiton
\newcounter{dfn}[section]\setcounter{dfn}{0}
\renewcommand{\thedfn}{\arabic{section}.\arabic{dfn}}
\newenvironment{dfn}[2][]{%
\refstepcounter{dfn}%
\ifstrempty{#1}%
{\mdfsetup{%
frametitle={%
\tikz[baseline=(current bounding box.east),outer sep=0pt]
\node[anchor=east,rectangle,fill=cyan!20]
{\strut Definition~\thedfn};}}
}%
{\mdfsetup{%
frametitle={%
\tikz[baseline=(current bounding box.east),outer sep=0pt]
\node[anchor=east,rectangle,fill=cyan!20]
{\strut Definition~\thedfn:~#1};}}%
}%
\mdfsetup{innertopmargin=5pt,innerbottommargin=10pt,linecolor=cyan!20,%
linewidth=2pt,topline=true,%
frametitleaboveskip=\dimexpr-\ht\strutbox\relax
}
\begin{mdframed}[]\relax%
\label{#2}}{\end{mdframed}}

%%%%%%%%%%%%%%%%%%%%%%%%%%%%%%
%Information
\newcounter{info}[section]\setcounter{info}{0}
\renewcommand{\theinfo}{\arabic{section}.\arabic{info}}
\newenvironment{info}[2][]{%
  \refstepcounter{info}%
  \ifstrempty{#1}%
  {\mdfsetup{%
  frametitle={%
  \tikz[baseline=(current bounding box.east),outer sep=0pt]
  \node[anchor=east,rectangle,fill=gray!20]
  {\strut Information~\theinfo};}}
  }%
  {\mdfsetup{%
  frametitle={%
  \tikz[baseline=(current bounding box.east),outer sep=0pt]
  \node[anchor=east,rectangle,fill=gray!20]
  {\strut Information~\theinfo:~#1};}}%
  }%
  \mdfsetup{innertopmargin=5pt,innerbottommargin=10pt,linecolor=gray!20,%
  linewidth=2pt,topline=true,%
  frametitleaboveskip=\dimexpr-\ht\strutbox\relax
  }
  \begin{mdframed}[]\relax%
  \label{#2}
}
{
  \end{mdframed}
}

%%%%%%%%%%%%%%%%%%%%%%%%%%%%%%
%Example
\newcounter{eg}[section]\setcounter{eg}{0}
\renewcommand{\theeg}{\arabic{section}.\arabic{eg}}
\newenvironment{eg}[2][]{%
  \refstepcounter{eg}%
  \ifstrempty{#1}%
  {\mdfsetup{%
  frametitle={%
  \tikz[baseline=(current bounding box.east),outer sep=0pt]
  \node[anchor=east,rectangle,fill=black!20]
  {\strut Example~\theeg};}}
  }%
  {\mdfsetup{%
  frametitle={%
  \tikz[baseline=(current bounding box.east),outer sep=0pt]
  \node[anchor=east,rectangle,fill=black!20]
  {\strut Example~\theeg:~#1};}}%
  }%
  \mdfsetup{innertopmargin=5pt,innerbottommargin=10pt,linecolor=black!20,%
  linewidth=2pt,topline=true,%
  frametitleaboveskip=\dimexpr-\ht\strutbox\relax
  }
  \begin{mdframed}[]\relax%
  \label{#2}
}
{
  \end{mdframed}
}


\definecolor{dfn-color}{rgb}{0,0.48,0.65}

% https://tex.stackexchange.com/questions/227913/meet-and-join-symbols-for-mathematical-lattice-utf-∨-and-∧
\DeclareFontFamily{U}{matha}{\hyphenchar\font45}
\DeclareFontShape{U}{matha}{m}{n}{
      <5> <6> <7> <8> <9> <10> gen * matha
      <10.95> matha10 <12> <14.4> <17.28> <20.74> <24.88> matha12
      }{}
\DeclareSymbolFont{matha}{U}{matha}{m}{n}
\DeclareMathSymbol{\meet}{2}{matha}{"5E}
\DeclareMathSymbol{\join}{2}{matha}{"5F}

\newcommand*{\QED}{\hfill\ensuremath{\square}}
\DeclareMathOperator{\lcm}{lcm}


\begin{document}
\lecture{Class Discussion Notes}{February 13, 2019}{Dev Dabke}

\section{A View from Berkeley}
DW: Discuss conclusions, did people learn something new? End of the paper?
What was the conclusion of the paper?
Did we like the end?
Some autotuners were thrown in for good luck?

DW: If we go into the conclusion section.
The conclusion just rambles a little bit.
The weakest conclusion ever.
Terrible conclusion!
Conclusions should be concise and to the point, and this is not that.

Anything else?
Greg: can we talk about autotuners?

DW: what is an autotuner?
DW: an autotuner is a piece of software that has an optimization algorithm that looks at how code is running and then somehow modifies the code based on some optimization criterion.
It could just be changing parameters.
e.g.\ if you have the blocking size of an algorithm or some other parameter (outer loop vs. inner loop parallelism) that you want to optimize, but you change parameters as you run
Greg: not used widely?
DW: it depends. If you're doing supercomputing with dense matrix codes, a lot of the libraries that people use under the hood use that.
FFTW is by definition an autotuner.
It subsumes all implementions of FFT within it.
At run time, it tests the arch and tune the algo dynamically to the running code.
BLAS is something similar what they're doing these days what the supercomputing stuff people use.
LINPACK

Some autotuner optimize for arch, some optimize for dunamic events going on in the system.

DW: any other things that came up that you didn't know what they were?
Peter: how do virtual machines connect to parallel computing?
DW: it doesn't connect at all. Probably some random research thing that someone threw in for one of the professors.
DW: talking about RAMP is a non-sequitur.
``Academics shouldn't build chips'' but both Krste and Patterson went and build chips in academia.

\section{Matrix Computation}
DW: what is the second paper about?
Is it a paper?

Sam: it's a chapter from a book.

DW: what is the problem we're trying to solve.

Sam: Ax = b, find x

DW: what is A, x, and b
They present this is a system of linear equations.
What is A? Is it a scalar? A vector? A 1d or 2d structure?

Sam: 2d

DW: what does it represent?
Going back to our basic linear algebra here:
\begin{align*}
    a_{11} x_{1} + a_{12} x_{2} + \cdots + a_{1n} x_{n} &= b_{1} \\
    a_{21} x_{1} + a_{22} x_{2} + \cdots + a_{2n} x_{n} &= b_{2} \\
    & \vdots \\
    a_{n1} x_{1} + a_{n2} x_{2} + \cdots + a_{nn} x_{n} &= b_{n}
\end{align*}

DW: What does the \( A \) matrix represent?
Sam: The coefficients

DW: What is the \( b \)s?
It's a 1d vector of \( b \)s?
If we factor out all of the scalar, we can rewrite as
\[
A \vec{x} = \vec{b}
\]

DW: is this useful? Do we do this?
Yes it's useful and people do it all the time!

??: Take the inverse of the matrix
DD: Gaussian elimination

DW: What is GE?
We take each row and multiplied it by something and subtracted it from another row.
It all boils down to GE in some variation?
DW: runtime of GE?
?: \( O (n^3) \)
DW: each row elim is order \( O(n) \)
We could possibly have to do it \( O(n) \) times for a row to get it in order.
And there are \( O(n) \) rows.
Are we idiots because we did what the linear algebra book told us to do?
If we have \( n \) processors does that help?
???: it reduces the complexity to \( O(n^2) \).
We can do the multiply, add, and subtract in parallel.

What about \( n^2 \) processors?
It doesn't really help.

DW: Is this paper about GE? Heck no (maybe in a more abstract sense).

What is this paper about?

DW: iterative algorithms
What is the trick that's being played here?
Important part is that it's \textit{iterative}.

Take a look at the math trick that's played here:
boil everything down to this nice equation here:
\[
x_{i}(t + 1) = -\frac{1}{a_{ii}} [ \sum_{j \neq i} a_{ij} x_{j} (t) - b_{i} ]
\]
What's going on here?
They basically solve this for \( x_{i} \) in terms of all of the other \( x_{j} \)s.
Then they parameterize it with a time variable.
Jacobi Iteration.
Why is this good?
??: It can be done in parallel. The solution for each \( x_{i} \) can be computing in parallel.
DW: what is this dependent on?
Not so great, it's dependent on everything else.
We need to look at all other \( x_{j} \)s and we have to multiply and addition (and subtraction of a constant).
DW: how many times do we have to iterate?
R?: depends on word length
DW: does this actually converge to the solution?
DW: it always converges if there is a solution
Some other algorithms will converge faster.
Not guaranteed to get closed on each step, i.e.\ divergence for a few steps doens't mean anything.
\\ \\
DW: If we look at the equation, we can execute everything in parallel.
DW: Can we do better?
We can parallelize the multiplies.
On a sequential computer, we have \( O (n^2) \) assuming time is constant.

With \( O(n) \) processors, we have \( O(n) \) running time.

With \( O(n^2) \) processors, we can do multiply in parallel, but we have to do additions, we this brings us down to \( \log{n} \).
Communication pattern is all-to-all, which is pretty terrible.
\\ \\
Gauss-Siedel, is this good or bad for a parallel computer.
Terrible for a parallel computer!
We could break the dependence by recomputing another matrix, but it reduces to Jacobi iteration.
It converges faster, but worse for parallel arch.
\\ \\
JOR: weighting the residual and the current value.
Happens to converge faster, not trivial as to why this is the case.
\\ \\
Later in the paper, they come up with a practical problem.
DW: What can you use these to solve?
You can use it to solve differential equations.
What can this be applied to?
e.g.\ sheet of metal and discretize the space [Diagram 1 - of grid]
compute the temperature
Is this better or worse from before?
DD: same problem, but linearly sparse, because you only need a fixed constant of values [Diagram 2 - block matrix]
So that's kinda neat, this is very useful.
It's \( O(n^2) \).
With \( O(n) \) processors, we get down to \( O(n) \).
With \( O(n^2) \) processors, we get down to \( O(1) \).
And the communication pattern is much easier!
\\ \\
Using gray code to map communication topology for the multigrid, but you can use different refinements of the mesh.

\section{Weather Code}
Weather yesterday: snow, rain, freezing rain, sleet, hail, lots of stuff.
Does this paper for today tell us about ML? Nope.
Do you need ML to predict the weather?
Is the weather dependent on basic laws of physics?
This paper is trying to use basic laws of physics to predict the weather.
DW: do people still use this?
Greg/Dev: yes, both this (and maybe ML).
DW: GFDL moved their supercomputers from Princeton to Oak Ridge, but they might move them back.
DW: they are still using things that basically look like this.
\\ \\
DW: How is this weather code structured?
DW: it takes the room and segments it.
Read flatland, not PC in 2019, but an interesting read.
The 3D-segments approximate cubes.
We think we live in 3-space, so might as well segment into 3D objects.
DW: communication pattern is with neighbors.
What are the parameters?
R?: wind vector, temp, specific humidity, pressure, geopotential, pressure.
DW: we don't take into consideration if its raining in a cube in the model, but you can just take it as a function of these other factors.
They take the cubes and map it to some cylindrical formation.
Newer models do try to have poles and other more accurate notions.
DW: can we discretize time and space?
They do it.
We compute the weather one timestep in the future based on some notion of step size.
What is the communication pattern?
Compute from neighborhood, 7 nodes (self, left, right, front, back, up, down).
\\ \\
How would we map this onto a computer?
Sequential is \( O(n^3) \).
The edges are fixed with boundary conditions.
Some of this has changed over time, e.g.\ ground temperature.
With \( O(n^3) \) processors, we can reach \( O(1) \).
What does the communication look like?
\\ \\
With \( O(n) \) processors, with just planes, we have to communicate \( O(n^2) \) values.
etc.
\\ \\
In L1, a ``sphere'' i.e.\ a cube maximizes volume while minimizing the surface area for communication.
\\ \\
What's the communication pattern for the tube?
We have to allow for communication around.
Thinking about mapping, we have to think about mapping to an actual machine.
In the paper, they map this to the NYU ultracomputer.
NYU used to do arch.
Ultracomputer had a butterfly communication pattern.
They say that the weather code doesn't do particularly well on the ultracomputer because there is no good parallelization that maps to a butterfly network.
You want to try to map the communication pattern to the problem, otherwise you could have a problem.
\\ \\
Can you start before everyone else is done?
Code is written with a heavyweight barrier.
Ultracomputer is a shared mem machine.
You could have a 3D matrix.
\\ \\
DW: want to talk about three terms (nomenclature).
\begin{enumerate}
    \item \textit{speed up}: the time it takes for a sequential computer divided by the time it takes for a parallel computer.
    \item \textit{scaling}: plot performance against (y-var) against number of resources (x-var); if something ``scales,'' as you add more resources, then the performance goes up.
    \item \textit{strong scaling}: fixed problem size, more resources, still scales; problem size / core goes down, but still scales
    \item \textit{weak scaling}: you need to increase the problem size, then scales; we keep the work the same per core
\end{enumerate}


\end{document}
