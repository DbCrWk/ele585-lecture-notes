\documentclass[12pt]{article}

\usepackage[a4paper,margin=2cm]{geometry}
\usepackage{amsmath}
\usepackage{amssymb}
\usepackage{amsthm}
\usepackage{amsfonts}

\usepackage{algpseudocode}
\usepackage[plain]{algorithm}

\usepackage{caption}
\usepackage{graphicx}
\usepackage{float}
\usepackage{listings}
\usepackage[table]{xcolor}
\usepackage{booktabs}
\usepackage[colorlinks,allcolors=newblue]{hyperref}
\usepackage{microtype}

\usepackage{tikz}
\usepackage[framemethod=TikZ]{mdframed}
\definecolor{newblue}{rgb}{0.2,0.2,0.6}

\usepackage{listings}
\lstset{
    basicstyle=\scriptsize\ttfamily,
    commentstyle=\ttfamily\color{gray},
    numbers=left,
    numberstyle=\ttfamily\color{gray}\footnotesize,
    stepnumber=1,
    numbersep=5pt,
    backgroundcolor=\color{white},
    showspaces=false,
    showstringspaces=false,
    showtabs=false,
    frame=single,
    tabsize=2,
    captionpos=b,
    breaklines=true,
    breakatwhitespace=false,
    title=\ttfamily\lstname,
    escapeinside={},
    keywordstyle={},
    morekeywords={}
}

\newcommand{\handout}[5]{
   \renewcommand{\thepage}{#1-\arabic{page}}
   \noindent
   \begin{center}
   \framebox{
      \vbox{
    \hbox to 5.78in { {\bf ELE 585: Parallel Computing} \hfill #2 }
       \vspace{4mm}
       \hbox to 5.78in { {\Large \hfill #5  \hfill} }
       \vspace{2mm}
       \hbox to 5.78in { {\it #3 \hfill #4} }
      }
   }
   \end{center}
   \vspace*{4mm}
}

\renewcommand{\paragraph}[1]{\medskip \noindent {\bf #1.}}

\newcommand{\lecture}[3]{\handout{#1}{#2}{Professor David Wentzlaff}{By #3}{#1}}

%%%%%% FOUND AT
% https://texblog.org/2015/09/30/fancy-boxes-for-theorem-lemma-and-proof-with-mdframed/

%%%%%%%%%%%%%%%%%%%%%%%%%%%%%%
%Theorem
\newcounter{theo}[section]\setcounter{theo}{0}
\renewcommand{\thetheo}{\arabic{section}.\arabic{theo}}
\newenvironment{theo}[2][]{%
\refstepcounter{theo}%
\ifstrempty{#1}%
{\mdfsetup{%
frametitle={%
\tikz[baseline=(current bounding box.east),outer sep=0pt]
\node[anchor=east,rectangle,fill=blue!20]
{\strut Theorem~\thetheo};}}
}%
{\mdfsetup{%
frametitle={%
\tikz[baseline=(current bounding box.east),outer sep=0pt]
\node[anchor=east,rectangle,fill=blue!20]
{\strut Theorem~\thetheo:~#1};}}%
}%
\mdfsetup{innertopmargin=5pt,innerbottommargin=10pt,linecolor=blue!20,%
linewidth=2pt,topline=true,%
frametitleaboveskip=\dimexpr-\ht\strutbox\relax
}
\begin{mdframed}[]\relax%
\label{#2}}{\end{mdframed}}

%%%%%%%%%%%%%%%%%%%%%%%%%%%%%%
%Proposition
\newcounter{prp}[section]\setcounter{prp}{0}
\renewcommand{\theprp}{\arabic{section}.\arabic{prp}}
\newenvironment{prp}[2][]{%
\refstepcounter{prp}%
\ifstrempty{#1}%
{\mdfsetup{%
frametitle={%
\tikz[baseline=(current bounding box.east),outer sep=0pt]
\node[anchor=east,rectangle,fill=magenta!20]
{\strut Proposition~\theprp};}}
}%
{\mdfsetup{%
frametitle={%
\tikz[baseline=(current bounding box.east),outer sep=0pt]
\node[anchor=east,rectangle,fill=magenta!20]
{\strut Proposition~\theprp:~#1};}}%
}%
\mdfsetup{innertopmargin=5pt,innerbottommargin=10pt,linecolor=magenta!20,%
linewidth=2pt,topline=true,%
frametitleaboveskip=\dimexpr-\ht\strutbox\relax
}
\begin{mdframed}[]\relax%
\label{#2}}{\end{mdframed}}

%%%%%%%%%%%%%%%%%%%%%%%%%%%%%%
%Lemma
\newcounter{lem}[section]\setcounter{lem}{0}
\renewcommand{\thelem}{\arabic{section}.\arabic{lem}}
\newenvironment{lem}[2][]{%
\refstepcounter{lem}%
\ifstrempty{#1}%
{\mdfsetup{%
frametitle={%
\tikz[baseline=(current bounding box.east),outer sep=0pt]
\node[anchor=east,rectangle,fill=green!20]
{\strut Lemma~\thelem};}}
}%
{\mdfsetup{%
frametitle={%
\tikz[baseline=(current bounding box.east),outer sep=0pt]
\node[anchor=east,rectangle,fill=green!20]
{\strut Lemma~\thetheo:~#1};}}%
}%
\mdfsetup{innertopmargin=5pt,innerbottommargin=10pt,linecolor=green!20,%
linewidth=2pt,topline=true,%
frametitleaboveskip=\dimexpr-\ht\strutbox\relax
}
\begin{mdframed}[]\relax%
\label{#2}}{\end{mdframed}}

%%%%%%%%%%%%%%%%%%%%%%%%%%%%%%
%Proof

\newcounter{prf}[section]\setcounter{prf}{0}
\renewcommand{\theprf}{\arabic{section}.\arabic{prf}}
\newenvironment{prf}[2][]{%
\refstepcounter{prf}%
\ifstrempty{#1}%
{\mdfsetup{%
frametitle={%
\tikz[baseline=(current bounding box.east),outer sep=0pt]
\node[anchor=east,rectangle,fill=red!20]
{\strut Proof~\theprf};}}
}%
{\mdfsetup{%
frametitle={%
\tikz[baseline=(current bounding box.east),outer sep=0pt]
\node[anchor=east,rectangle,fill=red!20]
{\strut Proof~\theprf:~#1};}}%
}%
\mdfsetup{innertopmargin=5pt,innerbottommargin=10pt,linecolor=red!20,%
linewidth=2pt,topline=true,%
frametitleaboveskip=\dimexpr-\ht\strutbox\relax
}
\begin{mdframed}[]\relax%
\label{#2}}{\end{mdframed}}

%%%%%%%%%%%%%%%%%%%%%%%%%%%%%%
%Remark
\newcounter{rmk}[section]\setcounter{rmk}{0}
\renewcommand{\thermk}{\arabic{section}.\arabic{rmk}}
\newenvironment{rmk}[2][]{%
\refstepcounter{rmk}%
\ifstrempty{#1}%
{\mdfsetup{%
frametitle={%
\tikz[baseline=(current bounding box.east),outer sep=0pt]
\node[anchor=east,rectangle,fill=yellow!50]
{\strut Remark~\thermk};}}
}%
{\mdfsetup{%
frametitle={%
\tikz[baseline=(current bounding box.east),outer sep=0pt]
\node[anchor=east,rectangle,fill=yellow!50]
{\strut Remark~\thermk:~#1};}}%
}%
\mdfsetup{innertopmargin=5pt,innerbottommargin=10pt,linecolor=yellow!50,%
linewidth=2pt,topline=true,%
frametitleaboveskip=\dimexpr-\ht\strutbox\relax
}
\begin{mdframed}[]\relax%
\label{#2}}{\end{mdframed}}

%%%%%%%%%%%%%%%%%%%%%%%%%%%%%%
%Definiton
\newcounter{dfn}[section]\setcounter{dfn}{0}
\renewcommand{\thedfn}{\arabic{section}.\arabic{dfn}}
\newenvironment{dfn}[2][]{%
\refstepcounter{dfn}%
\ifstrempty{#1}%
{\mdfsetup{%
frametitle={%
\tikz[baseline=(current bounding box.east),outer sep=0pt]
\node[anchor=east,rectangle,fill=cyan!20]
{\strut Definition~\thedfn};}}
}%
{\mdfsetup{%
frametitle={%
\tikz[baseline=(current bounding box.east),outer sep=0pt]
\node[anchor=east,rectangle,fill=cyan!20]
{\strut Definition~\thedfn:~#1};}}%
}%
\mdfsetup{innertopmargin=5pt,innerbottommargin=10pt,linecolor=cyan!20,%
linewidth=2pt,topline=true,%
frametitleaboveskip=\dimexpr-\ht\strutbox\relax
}
\begin{mdframed}[]\relax%
\label{#2}}{\end{mdframed}}

%%%%%%%%%%%%%%%%%%%%%%%%%%%%%%
%Information
\newcounter{info}[section]\setcounter{info}{0}
\renewcommand{\theinfo}{\arabic{section}.\arabic{info}}
\newenvironment{info}[2][]{%
  \refstepcounter{info}%
  \ifstrempty{#1}%
  {\mdfsetup{%
  frametitle={%
  \tikz[baseline=(current bounding box.east),outer sep=0pt]
  \node[anchor=east,rectangle,fill=gray!20]
  {\strut Information~\theinfo};}}
  }%
  {\mdfsetup{%
  frametitle={%
  \tikz[baseline=(current bounding box.east),outer sep=0pt]
  \node[anchor=east,rectangle,fill=gray!20]
  {\strut Information~\theinfo:~#1};}}%
  }%
  \mdfsetup{innertopmargin=5pt,innerbottommargin=10pt,linecolor=gray!20,%
  linewidth=2pt,topline=true,%
  frametitleaboveskip=\dimexpr-\ht\strutbox\relax
  }
  \begin{mdframed}[]\relax%
  \label{#2}
}
{
  \end{mdframed}
}

%%%%%%%%%%%%%%%%%%%%%%%%%%%%%%
%Example
\newcounter{eg}[section]\setcounter{eg}{0}
\renewcommand{\theeg}{\arabic{section}.\arabic{eg}}
\newenvironment{eg}[2][]{%
  \refstepcounter{eg}%
  \ifstrempty{#1}%
  {\mdfsetup{%
  frametitle={%
  \tikz[baseline=(current bounding box.east),outer sep=0pt]
  \node[anchor=east,rectangle,fill=black!20]
  {\strut Example~\theeg};}}
  }%
  {\mdfsetup{%
  frametitle={%
  \tikz[baseline=(current bounding box.east),outer sep=0pt]
  \node[anchor=east,rectangle,fill=black!20]
  {\strut Example~\theeg:~#1};}}%
  }%
  \mdfsetup{innertopmargin=5pt,innerbottommargin=10pt,linecolor=black!20,%
  linewidth=2pt,topline=true,%
  frametitleaboveskip=\dimexpr-\ht\strutbox\relax
  }
  \begin{mdframed}[]\relax%
  \label{#2}
}
{
  \end{mdframed}
}


\definecolor{dfn-color}{rgb}{0,0.48,0.65}

% https://tex.stackexchange.com/questions/227913/meet-and-join-symbols-for-mathematical-lattice-utf-∨-and-∧
\DeclareFontFamily{U}{matha}{\hyphenchar\font45}
\DeclareFontShape{U}{matha}{m}{n}{
      <5> <6> <7> <8> <9> <10> gen * matha
      <10.95> matha10 <12> <14.4> <17.28> <20.74> <24.88> matha12
      }{}
\DeclareSymbolFont{matha}{U}{matha}{m}{n}
\DeclareMathSymbol{\meet}{2}{matha}{"5E}
\DeclareMathSymbol{\join}{2}{matha}{"5F}

\newcommand*{\QED}{\hfill\ensuremath{\square}}
\DeclareMathOperator{\lcm}{lcm}


\begin{document}
\lecture{Class Discussion Notes}{February 13, 2019}{Dev Dabke, Samuel Hsia}

\section{A View from Berkeley}

\subsection{On the Conclusion}

The general consensus was that the conclusion rambled on quite a bit. Because, of this, we agreed that, in terms of position papers, it is rather weak. Conclusions should be concise and to the point -- this paper fails to achieve this. 

\subsection{Final Remarks}

\subsubsection{Autotuners}

An \textbf{autotuner}\footnote{Autotuners are also referred to as "self-tuning systems" in the context of control theory based applications.} is a piece of software with an optimization algorithm that looks at how code is running and then subsequently modifies the code based on some optimization criterion.

Autotuners often accomplish this modification by changing parameters. For example, if you have the blocking size of an algorithm or some other parameter (outer loop vs. inner loop parallelism) that you want to optimize, you can change these parameters as you execute your program. In general, these systems attempt to find an optimal condition for an objective function while minimizing an associated error function.

Autotuners are used widely -- but only in certain contexts. In the world of supercomputing, where dense matrix codes are prevalent, a lot of the computational libraries that people use under the hood rely on autotuners. An example of this would be \textbf{FFTW (Fastest Fourier Transform in the West}. It subsumes all implementations of FFT within it and, at run time, tests the architecture and tunes the algorithm dynamically to the running code. While the FFTW optimizes based on the architecture, other autotuner will optimize based on the dynamic events of the system. Other more notable software libraries that utilize autotuners include BLAS and LINPACK (both heavily relied upon in supercomputing applications).

\subsubsection{Virtual Machines}

Virtual machines and parallel computing don't really have any direct connections. It is very possible that the discussion on virtual machines was included for one of the authors.

\subsubsection{RAMP}

The discussion on RAMP is rather non-sequitur. As proven later on, the sentiment that ``Academics shouldn't build chips'' was not really followed as both Krste Asanovic and David Patterson went on and built prototype chips in academia.

\section{Iterative Methods for Linear Equations}

This is not a papers - but rather a chapter within a book\footnote{\textit{Parallel and Distributed Computing Numerical Methods} by Bertsekas and Tsitsikus.}. It focuses on solving the classic system of linear equations:

\begin{align*}
    \text{Given }Ax = b, \text{find } x
\end{align*}

where $A$ is a 2D structure containing the coefficients of the system and $b$ is a 1D vector.

Going back to our basic linear algebra here, our problem can be expanded to:
\begin{align*}
    a_{11} x_{1} + a_{12} x_{2} + \cdots + a_{1n} x_{n} &= b_{1} \\
    a_{21} x_{1} + a_{22} x_{2} + \cdots + a_{2n} x_{n} &= b_{2} \\
    & \vdots \\
    a_{n1} x_{1} + a_{n2} x_{2} + \cdots + a_{nn} x_{n} &= b_{n}
\end{align*}

If we factor out all of the scalar, we can rewrite as
\[
A \vec{x} = \vec{b}
\]

\textbf{Is this useful and how do we get about solving this problem?}

Linear systems are extremely powerful and the canonical way of solving this is by finding $A^{-1}$ with \textbf{Gaussian Elimination}.

\subsection{Gaussian Elimination}

In general, we take each row, multiply it by a constant, and subtract it from another row to end up with a diagonal matrix. 

The runtime of Gaussian Elimination is \( O (n^3) \):
\begin{itemize}
    \item Each row elimination is order \( O(n) \)
    \item We could have to do it \( O(n) \) times for a row to get it in order.
    \item There are \( O(n) \) rows.
\end{itemize}

This runtime complexity is extremely bad but if we have \( n \) processors that reduces the complexity to \( O(n^2) \) because we can do the multiply, add, and subtract in parallel. However, scaling to \( n^2 \) processors does not really help. \textbf{However, this paper is not about Gaussian Elimination} -- it is about \textbf{iterative algorithms}. 

\subsection{Jacobi Algorithm}

We can reformulate the problem based on an iterative approach and boil everything down to this nice equation\footnote{This reformulation is known as the \textbf{Jacobi Algorithm}}:

\[
x_{i}(t + 1) = -\frac{1}{a_{ii}} [ \sum_{j \neq i} a_{ij} x_{j} (t) - b_{i} ]
\]

where you solve for \( x_{i} \) in terms of all of the other \( x_{j} \)s. The key operations are  multiplication and addition (and subtraction of a constant).

One thing that we are garuanteed is convergence -- as long as a solution to the linear system of equations exists. Between different algorithms, some will lead to a faster convergence. It's also important to note that we are not guaranteed to get closer to the solution on each iteration, (i.e.\ divergence for a few steps doesn't mean much).

Note the parameterization of this problem with a time variable -- this allows the problem to be tackled in a parallel manner (i.e. the solution for each \( x_{i}(t+1) \) can be computing in parallel). However, in calculating \( x_{i}(t+1) \), we need to look at all other \( x_{j} (t)\)s -- this causes a huge dependency bottleneck. 

Assuming time is constant, we can achieve a \( O (n^2) \) runtime on a sequential computer. If we can parallelize the multiplies With \( O(n) \) processors, we have \( O(n) \) running time. With \( O(n^2) \) processors, we can do multiplies in parallel, but we would still have to do additions -- this brings us down to \( \log{n} \).

Also, communication pattern is all-to-all, which is pretty terrible.

\subsection{Gauss-Seidel Algorithm}

The Gauss-Seidel algorithm is terrible for a parallel computer\footnote{We could break the dependence by recomputing another matrix, but it reduces to Jacobi iteration.}! While this algorithm converges faster, it is worse for parallel architectures.

\subsection{Jacobi Overrelaxation (JOR)}
The intuition behind this is that we are weighting the residual and the current value to achieve a faster convergence (not trivial as to why this is the case).

\subsection{Application examples}

Later in the paper, we talk about using these iterative methods to solve differential equations (e.g.\ discretizing the space and calculating the temperature of a sheet of metal [Diagram 1 - of grid]).

This is, in general, the same problem as our aforementioned linear system of equations except for the fact that it is linearly sparse. Thus, you only need a fixed constant of values [Diagram 2 - block matrix].

Because of this sparse nature, it has a \( O(n^2) \) runtime complexity. With \( O(n) \) processors, we can get that down to \( O(n) \). With \( O(n^2) \) processors, we get down to \( O(1) \) -- something that we could not accomplish before! On top of that, the communication pattern is much easier!

Lastly, we can use gray code to map the communication topology for a multigrid topology.

\section{The Weather Code}
Weather yesterday: snow, rain, freezing rain, sleet, hail, lots of stuff.
Does this paper for today tell us about ML? Nope.
Do you need ML to predict the weather?
Is the weather dependent on basic laws of physics?
This paper is trying to use basic laws of physics to predict the weather.
DW: do people still use this?
Greg/Dev: yes, both this (and maybe ML).
DW: GFDL moved their supercomputers from Princeton to Oak Ridge, but they might move them back.
DW: they are still using things that basically look like this.
\\ \\
DW: How is this weather code structured?
DW: it takes the room and segments it.
Read flatland, not PC in 2019, but an interesting read.
The 3D-segments approximate cubes.
We think we live in 3-space, so might as well segment into 3D objects.
DW: communication pattern is with neighbors.
What are the parameters?
R?: wind vector, temp, specific humidity, pressure, geopotential, pressure.
DW: we don't take into consideration if its raining in a cube in the model, but you can just take it as a function of these other factors.
They take the cubes and map it to some cylindrical formation.
Newer models do try to have poles and other more accurate notions.
DW: can we discretize time and space?
They do it.
We compute the weather one timestep in the future based on some notion of step size.
What is the communication pattern?
Compute from neighborhood, 7 nodes (self, left, right, front, back, up, down).
\\ \\
How would we map this onto a computer?
Sequential is \( O(n^3) \).
The edges are fixed with boundary conditions.
Some of this has changed over time, e.g.\ ground temperature.
With \( O(n^3) \) processors, we can reach \( O(1) \).
What does the communication look like?
\\ \\
With \( O(n) \) processors, with just planes, we have to communicate \( O(n^2) \) values.
etc.
\\ \\
In L1, a ``sphere'' i.e.\ a cube maximizes volume while minimizing the surface area for communication.
\\ \\
What's the communication pattern for the tube?
We have to allow for communication around.
Thinking about mapping, we have to think about mapping to an actual machine.
In the paper, they map this to the NYU ultracomputer.
NYU used to do arch.
Ultracomputer had a butterfly communication pattern.
They say that the weather code doesn't do particularly well on the ultracomputer because there is no good parallelization that maps to a butterfly network.
You want to try to map the communication pattern to the problem, otherwise you could have a problem.
\\ \\
Can you start before everyone else is done?
Code is written with a heavyweight barrier.
Ultracomputer is a shared mem machine.
You could have a 3D matrix.
\\ \\
DW: want to talk about three terms (nomenclature).
\begin{enumerate}
    \item \textit{speed up}: the time it takes for a sequential computer divided by the time it takes for a parallel computer.
    \item \textit{scaling}: plot performance against (y-var) against number of resources (x-var); if something ``scales,'' as you add more resources, then the performance goes up.
    \item \textit{strong scaling}: fixed problem size, more resources, still scales; problem size / core goes down, but still scales
    \item \textit{weak scaling}: you need to increase the problem size, then scales; we keep the work the same per core
\end{enumerate}


\end{document}