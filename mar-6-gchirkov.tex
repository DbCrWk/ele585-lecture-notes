
\documentclass[12pt]{article}

\usepackage[a4paper,margin=2cm]{geometry}
\usepackage{amsmath}
\usepackage{amssymb}
\usepackage{amsthm}
\usepackage{amsfonts}

\usepackage{algpseudocode}
\usepackage[plain]{algorithm}

\usepackage{caption}
\usepackage{graphicx}
\usepackage{float}
\usepackage{listings}
\usepackage[table]{xcolor}
\usepackage{booktabs}
\usepackage[colorlinks,allcolors=newblue]{hyperref}
\usepackage{microtype}

\usepackage{tikz}
\usepackage[framemethod=TikZ]{mdframed}
\definecolor{newblue}{rgb}{0.2,0.2,0.6}

\usepackage{listings}
\lstset{
    basicstyle=\scriptsize\ttfamily,
    commentstyle=\ttfamily\color{gray},
    numbers=left,
    numberstyle=\ttfamily\color{gray}\footnotesize,
    stepnumber=1,
    numbersep=5pt,
    backgroundcolor=\color{white},
    showspaces=false,
    showstringspaces=false,
    showtabs=false,
    frame=single,
    tabsize=2,
    captionpos=b,
    breaklines=true,
    breakatwhitespace=false,
    title=\ttfamily\lstname,
    escapeinside={},
    keywordstyle={},
    morekeywords={}
}

\newcommand{\handout}[5]{
   \renewcommand{\thepage}{#1-\arabic{page}}
   \noindent
   \begin{center}
   \framebox{
      \vbox{
    \hbox to 5.78in { {\bf ELE 585: Parallel Computing} \hfill #2 }
       \vspace{4mm}
       \hbox to 5.78in { {\Large \hfill #5  \hfill} }
       \vspace{2mm}
       \hbox to 5.78in { {\it #3 \hfill #4} }
      }
   }
   \end{center}
   \vspace*{4mm}
}

\renewcommand{\paragraph}[1]{\medskip \noindent {\bf #1.}}

\newcommand{\lecture}[3]{\handout{#1}{#2}{Professor David Wentzlaff}{By #3}{#1}}

\input{config/math-boxes}

\definecolor{dfn-color}{rgb}{0,0.48,0.65}

% https://tex.stackexchange.com/questions/227913/meet-and-join-symbols-for-mathematical-lattice-utf-∨-and-∧
\DeclareFontFamily{U}{matha}{\hyphenchar\font45}
\DeclareFontShape{U}{matha}{m}{n}{
      <5> <6> <7> <8> <9> <10> gen * matha
      <10.95> matha10 <12> <14.4> <17.28> <20.74> <24.88> matha12
      }{}
\DeclareSymbolFont{matha}{U}{matha}{m}{n}
\DeclareMathSymbol{\meet}{2}{matha}{"5E}
\DeclareMathSymbol{\join}{2}{matha}{"5F}

\newcommand*{\QED}{\hfill\ensuremath{\square}}
\DeclareMathOperator{\lcm}{lcm}


\begin{document}
\lecture{Class Discussion Notes}{March 6, 2019}{Grigory Chirkov}

\section{Snoopy Protocols}

Without caches – no problems with coherence
If scaling number of cores with fixed mem bw – relative bw goes down as 1/p
If we add caches, we get some problems
With write-back we have dirty copies of data in caches:
Problematic sequence
P1 changes data at addr0 from 27 to 5, keeps changes in local cache
P2 reads data from addr0 – reads 27 (after p1 wrote 25), is this a problem? Depends on your consistency model actually
Why do we like relaxed consistency models in 2019? There’s no general best consistency model. But most people want to have some preserving

Georgios: Maybe support weak consistencies and make compilers take care of ordering?
Dave: many points of view exist. Last year there was a hot battle around this topic, cause riscv guys are now elaborating consistency models for their architecture. Should riscv have TSO or allow weaker consistency models?

Grigory: can we support both weak load/store ordering and TSO?
Dave: yes, some people do this, but with you can easily make TSO out of weak consistency with fences in sw

Georgios: How much harder is strong consistency compared to weak/no consistency?
Dave: smth…

The sequence above doesn’t work for write through cache
With write through?
Cpu2: load 27
Cpu1: store 25, goes directly into memory
Cpu2: load, read 25 from cache, when irl it should be 25

MSI solves this problem, it makes processors broadcast its intentions and addresses on the bus, which gives info to other processors and allows to invalidate the stale lines of data

Write-through vs write-back and Invalidation vs update are orthogonal directions

Sam: what is dual-ported cache?
Dave: *draws diagrams from 475*

Ok, let’s talk about software coherence
How can we do this?
Replace loads with messages to other processors


Sam: do people do this?
Dave: nope, hardware acceleration really helps to improve performance of coherence schemes

Other ways?
Grigory: Just make all loads/stores uncacheable
Dave: yep, I’ve done this in Tilera

Other ways?
Dave: We can do this on the level of page translation
But the big problem with this approach is that page sizes are typically much bigger than cache sizes, so false sharing occurs very often

Let’s talk about new papers
Both from ISCA, ’83 and ’84 and one cites another. Who are the authors?
Dave: what is the first paper about? What is the main contribution?
Rakshit: authors thought that it’s that if you use temporal locality you can reduce the bw
Dave: but the main contribution is that you put state information inside the cache per line and the ability to update that state due to snoopy bus, no one has ever done this before


\end{document}
